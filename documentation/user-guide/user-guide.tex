%** File: user-guide.tex
%** Contains: User guide for the O-Plan system
%** Author: Brian Drabble (bd@rousay)
%** Created: Mon Jul 19 16:35:26 1993
%** Updated: Fri Jan 31 17:20:39 1997 by Jeff Dalton
%** Release Version: 3.1
%** Copyright: (c) 1996, AIAI, University of Edinburgh
%** This material may be reproduced by or for the U.S. Government pursuant
%** to the copyright license under the clause at DFARS 252.227-7032
%** (June 1975) -- Rights in Technical Data and Computer Software (Foreign).

\documentstyle [11pt,../oplan-doc]{article}
\projectdocument{3.1}{User Guide}
\created{Brian Drabble}{December 11, 1996}
\lastaltered{Jeff Dalton}{January 31, 1997}{17:51}

\parskip=6pt
\vfuzz=30pt
\hfuzz=30pt
\hyphenation{AutoCAD}

\begin{document}

\TitlePage
\Headers

\tableofcontents

\opsection{\protect\label{intro} Introduction}

\subsection{Contents of this Guide}

The aim of this guide is to provide the user with an overview of
setting up and running the O-Plan system. This guide assumes that
the O-Plan system has been successfully extracted from the tar image
file and installed and compiled correctly.

You should have installed O-Plan in a directory different from any
directory used for an earlier version of O-Plan to avoid any confusion
between files from different versions. The installation procedure is
described in the {\tt INSTALL} file included in the O-Plan
distribution. In the remainder of this document it will sometimes be
assumed that there is a shell command, {\tt oplan}, that can be used
to invoke O-Plan.  The {\tt INSTALL} file describes how to establish
such a command.

This guide is divided into the following sections:

\begin{enumerate}
\item {\bf Setting up to run O-Plan}: this section describes the O-Plan
startup script and the environment variables used to run the system. 
\item {\bf Running O-Plan}: this section describes how to run the
O-Plan system as a planning.
\item {\bf Plan and World Interfaces}: this section describes the
PlanWorld interface used by O-Plan to allow the user to examine on
either the plan or world state.
\item {\bf O-Plan Processes and Internal Structure}: this section
provides an overview of the way in which an O-Plan agent is
implemented in terms of the the mapping of {\em processes} to
components and the way in which these components communicate.
\item {\bf Typing Lisp Commands}: this section deals with inputting of
Lisp commands into the O-Plan system, e.g. to turn on some of the
debugging tools.
\item {\bf Dealing with Lisp Errors}: this sections deals with Lisp
errors which sometimes occur. These can either be caused by typing
errors in entering Lisp commands or by errors in the O-Plan
code. In the latter case they should be reported to the O-Plan
development team.
\item {\bf Control Panel and Trace Information within O-Plan}: this
section deals with obtaining different levels of trace information
within the O-Plan system. 
\item {\bf Example Demonstration}: this section provides step by step
instructions on running an example demonstration.
\end{enumerate}

\opsection{\protect\label{interfaces} Running O-Plan and its Interfaces}
The aim of this section is to describe the environment variables and
scripts which have been provided with the O-Plan system
together with details of the interface which allows for developer
access to the various components. 

\subsection{O-Plan Scripts}
A single script has been provided to start the O-Plan system and this
is as follows:

\begin{enumerate}
\item {\bf oplan} \newline
This provides the basic O-Plan system with a textual interface for
plan and world state browsing. 
\end{enumerate}

See Section~\ref{viewers} of this document for further details on the
Plan and World Viewers.

\subsection{\protect\label{variables}Environment Variables used by O-Plan}
In order to run the demonstrations the user should be in the X windows
environment and if this is not the case the the user should issue the
necessary commands to invoke it. In most cases, it is not necessary
for users to set any environment variables themselves.  However, the
user may set some of the variables in order to change the defaults if
that is desired.  The following sections describe the environment
variables which are used, their purpose and where necessary their
setting or defaults.

\begin{itemize}
\item {\bf setenv {\sc oplandir} $<$directory name$>$} \newline
The directory name should be the full name of the directory that
contains the O-Plan system.  This variable is set by the {\tt oplan}
script which should have been edited prior to the first invocation of
O-Plan as part of the installation procedure. The script's value
persists only for the duration of the run.  Users may also wish to set
this variable in their global environment as a shorthand way of
specifying the top level of the O-Plan file structure.  That is how it
will be used below.  However, that setting will have no effect on the
value used by {\tt oplan}.

\item {\bf setenv {\sc oplantmpdir} \${\sc home}/oplan-tmp} \newline
Specifies the directory into which temporary files generated by the
O-Plan system will be written.  It is recommended that this
environment variable be set to {\tt oplan-tmp} in the users
\${\sc home} directory for uniformity with other O-Plan installations.
Alternatively it may be:

\begin{enumerate}
\item set to any user specified directory to which the user has the necessary
read and write access privilege
\item left unspecified in which case all temporary files will be
written to the user's current working directory
\end{enumerate}
\end{itemize}

\subsection{\protect\label{scripts}Setting up to Run O-Plan}
The O-Plan system is invoked from the startup script {\tt oplan}
which is held in the {\tt {\sc \$oplandir}/bin} directory. There
are several ways to arrange for this script to be accessible as
a shell command.

\begin{enumerate}
\item Define a shell alias, for shells that provide aliases.
For instance, {\tt tcsh} users can add the following line to their
home directory's {\tt .tcshrc} file:

\begin{verbatim}
alias oplan "/applications/oplan/bin/oplan"
\end{verbatim}

Note that you do not have to use the name {\tt oplan} for the alias.

\item Alter your {\tt PATH} to include the directory containing the
script.
\item Move the script to a directory which is on the users current
{\tt PATH}.
\end{enumerate}

The O-Plan system is menu driven and requires the X Window System.

\opsection{\protect\label{running}Running O-Plan}
Once the setup steps have been carried out (defined in
subsections~\ref{variables} and \ref{scripts}) O-Plan can be invoked
by typing the relevant command.  For example:

\begin{verbatim}
   oplan
\end{verbatim}

When you start O-Plan, a number of windows should appear.  The first
window to appear ({\tt O-Plan Running Processes}) displays the names
of the currently running parts of O-Plan.  This window is described
later in this document in the Pseudo-processes section. After that,
one window will appear for the Task Assigner ({\sc ta}) and one for
each of the components of the planner ({\sc im}, {\sc am}, {\sc dm},
{\sc kp}). Finally a control panel ({\sc cp}) window will appear over
the top of the {\sc im} window. The window in which you typed the {\tt
oplan} command will stay connected to O-Plan. When all the other
windows have appeared, the prompt {\tt form>} should appear in this
window.  You can type Lisp commands in this window and consequently it
is called the Lisp interaction window.  For more information see the
section Typing Lisp Commands below.

The window configuration is specified by the {\tt -config} argument to
the {\tt oplan} command.  The default, screen-filling, configuration
can also be specified by starting O-Plan as follows:

\begin{verbatim} 
   oplan -config oplan-planner-default
\end{verbatim}

The planner will be brought up in its uninitialised state. This will
be indicated by a message at the top of the {\sc ta} window. The
message will also indicate the date the image was created as well as
the version number. 

Each of the separate parts of the planner has its own window which
allows the user to follow the plan as it is being generated. The major
parts of the O-Plan systems are described in the Section~\ref{major}.

\subsection{\protect\label{cla}O-Plan Command Line Arguments}
The {\tt oplan} startup script has been designed to accept a number of
optional arguments which allow the user to customise certain aspects
of the systems functions and display formats. A partial list of the
arguments are as follows:

\begin{tabular}{lp{11cm}}
Argument & Description \\
-break & Enters the Lisp break loop. This allows the user to examine
the state of the world before O-Plan starts up or to call O-Plan
procedures without starting O-Plan. \\
-config {\tt $<$filename$>$} & Use the indicated configuration file. A
series of configuration files are provided in {\sc \$oplandir}/{\tt lib}. \\
-eval {\tt $<$form$>$} & Evaluate {\tt $<$form$>$}, where {\tt
$<$form$>$} is a Common Lisp expression. \\
-load filename & Load the indicated file. \\
-noinit & Do not load any {\tt oplan-init} file. \\
-tfc $<$tfc-arg, ...$>$ & Run the {\sc tf} syntax checker rather than as
O-Plan. All arguments after the {\tt -tfc} will be processed by the
syntax checker. 
\end{tabular}

Note that all arguments are processed {\em before} O-Plan considers
loading an {\tt oplan-init} file. If you need something to happen
earlier, use {\tt -eval} or {\tt -load}.

\subsection{The oplan-init file}
When O-Plan starts, it looks for a file named {\tt oplan-init}, first
in the current directory and then in the user's home directory.  If
the file exists, is is loaded into Lisp.

The oplan-init file can contain (Lisp) commands that customise
certain aspects of O-Plan.  At present, only a few such
customisations are defined.  See the sections on defining a PostScript
viewer and on the whats-going-on file for details.

\subsection{Exiting O-Plan}
The standard way to exit is to use the Quit option in the Task
Assigner menu or to press the Quit button in the control panel
window.  However, if something has gone wrong, these methods may
not work.  In that case, you have several options.

The fastest way to exit is to type the quit character (usually control
and backslash) in the Lisp interaction window.  All the other windows
should vanish, and Lisp will exit, leaving you back at the shell.
This is a hard exit.  Lisp may not perform cleanup tasks such as
flushing buffered output.  Typing control and C in the Lisp
interaction window will also cause all other windows to vanish, but in
most cases Lisp will not exit.  To exit from Lucid Common Lisp, type
{\tt (lcl:quit)}.

\subsection{\protect\label{major}Major Components of the the
O-Plan System} 

The screen image in Figure~\ref{screen-dump} shows the screen layout
for the planner after the initial setup. O-Plan provides an
interface which allows for developer access to the various components.
These components are as follows:

\begin{figure}[htb]
\vspace{12.5cm}
\special{psfile=../shared/shared-screen-dev.ps}
\caption{Example screen image from the O-Plan System}
\label{screen-dump}
\end{figure}

\begin{itemize}
\item Task Assignment Window \newline
This window allows the user to give a range of top level commands,
such as giving the planner a new task, examining on certain aspects of
the plan once it has been generated.
\item Interface Manager \newline
The interface manager handles interactions with agents external to the
planner, such as the task assignment agent. The control panel is part
of the Interface Manager and is used to output changes in the levels
of diagnostics.
\item Agenda Manager \newline
The agenda manager keeps track of the current {\em issues} within the
plan and decides which issue to tackle next. A issue describes a
particular problem in the plan which needs to be addressed. For
example, expand an action to a more detailed level, make sure that a
condition of an action is satisfied at the right point, ensure enough
resources are available for an action, etc.
\item Database Manager \newline
The database manager provides the support routines required by the
knowledge sources and holds the emerging plan as it is being
generated.
\item Knowledge Source Platform \newline
All decision making within O-Plan is carried out through knowledge
sources. There is a separate knowledge source for each possible flaw
type within a plan and the invocation of the appropriate knowledge
sources is one of the functions of the agenda manager. Knowledge
sources run on a knowledge source platform. In the current system there
is only one such platform.
\item Running lights \newline
The running lights window shows the user which component is currently
active, and in the case of problems where the error may have occurred. 
\end{itemize}

The user should identify each of the above components of the screen as
they will be frequently referenced in the Example Demonstration
Section. 

\subsection{\protect\label{tfc}Compiling O-Plan Domains}
The input language used by the O-Plan system is Task Formalism ({\sc
tf}) which is used to describe the actions, their decomposition and
constraints, tasks, resources in the domain, etc. In order to ensure
that the {\em syntactic} description of a domain (in {\sc tf} format)
is correct a command has been provided. (This assumes the user has
defined an alias {\tt oplan} to invoke the O-Plan system. If an
alternative method has been defined the command should be altered
accordingly)

\begin{verbatim} 
   oplan -tfc
\end{verbatim}

The command will provide a trace of the compilation process and will
indicate via appropriate messages where syntax errors have occurred in
the encoding of a domain. A full description of the {\sc tf} language
can be found in the O-Plan {\sc tf} Manual.

\subsection{\protect\label{sanity}O-Plan Sanity Checker}
The O-Plan sanity checker can be run from the window in which the
O-Plan system was invoked. After start-up the window remains attached
to the O-Plan system and is capable of accepting Lisp expressions. 

The sanity checker should be run after a plan has been produced and
its function is to check that certain aspects of the plan are
syntactically correct. For example, all conditions have a contributor,
all actions which need expanding have been, etc. Any error reports
will appear in the Lisp interaction window. The number of errors will
be reported to the {\sc ta} window, and the user will need to type
return in the {\sc ta} window to continue. To run the sanity checker
type {\tt (sanity-check)} in the Lisp-interaction window.

\opsection{\protect\label{viewers}Plan and World Viewers}

In addition to the default text based viewer O-Plan also provides the
ability to view plans via a PostScript viewer.  One of the plan-view
options in the Task Assignment menu is {\tt Plan View}.  This causes a
PostScript picture of the current plan nodes and their order
relationships to be printed on a PostScript printer.  It is assumed
that the Unix command {\tt lpr} refers to such a printer by default.
(That is, {\tt lpr} is called without a {\tt -P} argument to select a
non-default printer.)

However, a number of PostScript viewers are available, and they allow
a user to display PostScript output on their screen rather than
waiting for a hardcopy.  It is possible to inform O-Plan of such a
viewer by defining the {\tt :ps-viewer} parameter in an {\tt
oplan-init} file.  For instance, to use GhostView, a user would place
the following line in their {\tt oplan-init} file:

\begin{verbatim} 
   (set-parameter :ps-viewer "ghostview -landscape")
\end{verbatim}

When this parameter has a value (the default is as above), a ``Viewer
one page'' option will appear in the ``Output format'' menu (the menu
that appears when a user selects {\tt Print graph} as the plan display
mode).  Selecting this option will cause the viewer to be run.  The
viewer is run in the background, and O-Plan can continue
independently.  However, O-Plan might not exit completely until the
viewer has exited as well.

If you have difficulty getting a viewer to work, it may help to know
that the viewer and {\tt lpr} are invoked using the Unix command

\begin{verbatim} 
   (viewer tmpfile ; /bin/rm tmpfile)&
\end{verbatim}

where viewer is, e.g., {\tt lpr} or {\tt ghostview -landscape} and
{\tt tmpfile} is the name of a temporary file generated by O-Plan.

\opsection{\protect\label{intervention}User Interaction with the Planner}
The O-Plan planner allows the user to intervene at any time during the
planning process to examine on the plan and world state and to set
certain modes which allow the user to take choices on behalf of the
planner. To intervene in the planning process the user needs to click
in the {\tt Intervene as User} button on the {\tt O-Plan Control
Panel}. As a result a menu will appear with the following options:

\begin{verbatim}
   Plan View
   World View
   Set Modes
   Break in
   Poison
   Quit
\end{verbatim}

Each of the options will now be described in further detail.

\begin{description}
\item[Plan View]: This requests a view of the Plan which can be
generated in several different formats:
\begin{itemize}
\item Text to screen: the plan will appear in the PlanWorld Viewer window.
\item Text to file: the plan will be sent to a user named file.
\item Print Graph: the plan can be sent to an appropriate print device.
\end{itemize}

\item[World View]:This requests a view of the World at a particular
point in the plan and can be generated in several different formats:
\begin{itemize}
\item Text to screen: the description will appear in the PlanWorld
Viewer window. 
\item Text to file: the description will be sent to a user named file.
\end{itemize}

\item[Set Modes]:
This allows the user to change the mode of {\em three} types of choice
in the planner. The modes can be set to either {\tt ask} or {\tt auto}
and can be changed by clicking on the option with the mouse. The modes
are as follows:

\begin{description}
\item[Binding mode] deals with the binding of specific values to
variables in the plan state.
\item[Schema selection mode] deals with the choice of schema for an
expansion of a node.
\item[Poison Handler mode] deals with the choice of an alternative
plan state should the current one prove fruitless.
\end{description}

When the planner encounters a choice in a mode set to {\tt ask} it
will prompt the user with an appropriate menu from which the user must
choose. The user is free to change mode as often as desired during the
planning process. 

\item[Break in]: This requests the planner to allow the user access to
the O-Plan system. When this option is chosen a window will appear as
follows: ({\tt Planner User (KS-USER)}

\begin{verbatim} 
>>Break: User Intervention

KS-USER-BREAK:
:C  0: Return from Break
    1: Return to pprocess scheduler

->
\end{verbatim}

The user is allowed to type any Lisp expression which can be
evaluated. The user must remember to return from the break by chosing
the appropriate option from the menu\footnote{The option and number
will vary according to the outcome of expressions typed by the user}.  

\item[Poison]: This requests the planner to poison the current plan
state and to provide the user with a list of alternative plan states
from which the user must choose one. The planner will then continue
the planning process from this new plan state. 
\end{description}

\opsection{O-Plan Processes and Internal Structure}
Most of the time, users do not have to know anything about how O-Plan
works internally.  However, a reasonably accurate model of how the
system works may make it easier to understand its behaviour,
especially when something goes wrong.  Since the processing model used
in version 3.1 may not be familiar to all users, this section provides
a brief description.  It also serves to explain the contents of the
``run-lights'' window.

An O-Plan agent, such as the planning or execution agents,
contains a number of components (Database Manager, Agenda Manager,
etc) that run in a semi-independent fashion and accomplish certain
tasks by sending each other messages.

O-Plan version 3.1 runs as a single invocation of Lisp as a single
Unix process.  Version 1 ran as several separate invocations of Lisp
which communicated via sockets.  Consequently, later versions are
sometimes referred to as the ``single-process'' version.  This has
lower communication overheads and is easier to manage, but the
appearance of independent operation is sometimes less complete.

In the single-process version of O-Plan, each component is
implemented as a pseudo process or p-process.  They are pseudo
because, unlike real processes, they don't represent an independent
thread of control.  They are more like event handlers.  An agent's
main loop calls each p-process in turn so long as there is a message
for it to handle.  The p-process usually handles one message and then
returns to the main loop.

When O-Plan is started, a small one-line window will appear before
all of the other windows.  This window is called the {\tt Running
Processes} window.  It displays the names of the parts of O-Plan that
are currently running or waiting for a running component to return a
value.  The window has the title ``O-Plan Running Processes'' and is
usually placed just below the Task Assignment window.

Ordinarily, only one p-process is active at once and they run
in turns in a round-robin fashion.  However, in some cases (such
as {\sc dm} requests), one p-process calls another as a subroutine.
In this case the calling p-process will be shown to the left of
the one it calls.  For instance, if the DM is running in the
ordinary way, a user will see:

\begin{verbatim} 
  +----------------------+
  |:DM                   |
  +----------------------+
\end{verbatim}

But if {\sc kp} calls the {\sc am} and the {\sc am} calls the {\sc dm}
before returning, a user will see:

\begin{verbatim} 
  +----------------------+
  |:KP :AM :DM           |
  +----------------------+
\end{verbatim}

In this case, only the {\sc dm} is actually running.  The {\sc am} is
waiting for the {\sc dm} to return some result, and {\sc kp} is
waiting for the {\sc am}.

Certain things cannot happen while the system is waiting for a
p-process to return.  In particular, no other p-processes can run.  So
if a button in pressed in the control panel, or an expression typed in
the Lisp interaction window, or a ``Plan View'' or ``World View'' selected
in the Task Assigner window, the requested effects may not happen
immediately.

Whenever O-Plan does not seem to be responding as expected when
something has been typed or selected from a menu entry, user should
look at the ``running processes'' to see what part of O-Plan is
currently running.  It is possible that the user may have forgotten to
answer a question or to select a menu entry, so that O-Plan is
waiting for the user to do that before going on.

\opsection{Typing Lisp Commands}
When starting O-Plan, the window in which the {\tt oplan} command
was typed does not return to the shell prompt.  Instead it functions
as a read-eval-print loop in which a user can type Lisp commands.  It
is also used for some low-level messages from O-Plan.  This window is
called the ``Lisp interaction window''.  Most of the time, a user can
ignore it, but may have to use it from time to time, e.g. to turn
on some of the debugging tools.  Any Lisp expression can be typed as a
command.

It's possible to type commands because there's a (pseudo-)process that
waits for input from this window and evaluates it when it comes.  When
a command is typed, there may therefore be a delay, until the next
time this pseudo-process gets to run, before the command is evaluated.

Note that if the input typed into the interaction window doesn't
amount to a complete Lisp expression, the whole system will eventually
stop and wait for the expression to be completed.  In particular, this
will happen if you {\tt <return>} is typed in the interaction window.
A user can see this has happened by looking at the one-line ``Running
processes'' window.  If nothing seems to be happening, but the
:LISP-LISTENER process is running, it may be because the user has
typed some incomplete input.

The listening p-process prints {\tt form>} as a prompt, but the
output that goes to the interaction window sometimes causes the prompt
to go off the top of the screen.

If a Lisp error occurs during the evaluation of a command that was
typed into this window, it is possible to get out of the error {\em
break loop}, and hence allow O-Plan to continue running, by using a
{\em restart} that has been set up for this purpose.  Look for the
option that says {\tt Return to O-Plan}.  A user can usually select
this option by typing {\tt :a} or {\tt 1}.  An example will make this
a bit clearer:

\begin{verbatim} 
   form> (not-a-function)
   >>Error: The function NOT-A-FUNCTION is undefined

   SYMBOL-FUNCTION:
      Required arg 0 (S): NOT-A-FUNCTION
   :C  0: Try evaluating #'NOT-A-FUNCTION again
       1: Return to O-Plan
       2: Return to pprocess scheduler

   -> :1
   Return to O-Plan
   NIL
   t
   form>
\end{verbatim}

\opsection{Dealing with Lisp Errors}
By ``a Lisp error'' we mean that O-Plan or the Lisp system invokes the
Lisp error signalling mechanism rather than, say, just printing a
message.  Unless the error is caused by an erroneous command typed
into the Lisp interaction window, it is usually due to a bug in
O-Plan and should be reported.

Your options for dealing with the Lisp error depend on what O-Plan
was doing when it occurred.

When a Lisp error occurs in the {\sc tf} compiler, a menu will appear
to make sure you notice the error.  There will be only one option,
``ok''.  Select ``ok'' to continue.  The TF compiler will exit and report
that the file contained an error.  That is, a Lisp error is treated as
if it were a single syntax error that applied to the whole {\sc tf}
file.

When the error occurs in some other part of the planner, there is
typically no neat way to recover from it, and a more complex menu
appears.  It will look something like this:

\begin{verbatim}
   Lisp error in process <process-name>
   ------------------------------------
   Allow the error
   Return to scheduler
   Force global reset
\end{verbatim}

{\tt <Process-name>} will be replaced by the name of the process that was
running when the error occurred: :AM, :DM, :KP, etc.  You should
look in the window associated with that process to see the error
message.  The meanings of the options are as follows:

\begin{description}
\item[Description] Allow Lisp to handle the error.  This will usually
result in a ``break loop'' as described below.  Select this option if
you plan to report the bug, because the bug report should include a
``backtrace'' (also described below).

\item[Return to scheduler] Exit the process in which the error
occurred so that the system can continue running.  In most cases, this
will not work very well, because the process in which the error
occurred needed to finish some task that was interrupted by the error.
This sort of recovery can also be attempted from a break loop, as
described below (again).

\item[Force global reset] Attempt to reinitialise O-Plan.  This will
abandon the attempt to find a plan (or whatever O-Plan was doing) and
start over as if you had selected the "Initialise planner" option in
the TA.  It is equivalent to evaluating the Lisp expression
(force-reset).
\end{description}

When the Lisp system handles an error, Lisp usually enters a ``break
loop'' in one of the O-Plan windows.  Ordinarily, you will see a menu
first, as described above.  In any case, you can tell that Lisp has
entered a break by the following signs:

\begin{enumerate}
\item O-Plan will stop producing any output.
\item The one-line ``running processes'' window will stop changing.
\item There will be an error message in a window associated with
the rightmost name in the ``running processes'' window.
\item The window will be displaying the ``>'' prompt.
\end{enumerate}

The first two can also happen when O-Plan is merely taking a long
time to do something; the others cannot.

In a break loop, the normal ways of exiting O-Plan by using the
control panel or the {\sc ta} menu will not work.  However, you can
type Lisp debugger commands and Lisp expressions in the the window,
after the ``$>$'' prompt.  Typing

\begin{verbatim}
  (exit-oplan)
\end{verbatim}

should cause O-Plan to exit.

\begin{verbatim}
  (force-reset)
\end{verbatim}

attempts to reset the system by wiping out all pending messages,
sending itself an {\tt :INIT} message, and returning to the process
scheduler.

If you're planing to report a bug, you should get a ``backtrace''
before exiting.  It shows the nested Lisp procedure calls that led up
to the error.  Get a backtrace by typing ``::b'' after the ``$>$''
prompt.  Grab the resulting output and the error message with the
mouse and put them both in the bug report.

In some cases you can (and may want to) get O-Plan to continue after
an error.  This is most likely to be so when the error is in a
relatively peripheral part of the planner, such as the {\sc pw}
viewer, but it can happen in other cases too.

In some cases, the Lisp system will offer you a reasonable way to
continue from the error, so always look at the options that appear
after the error message.

Another form of recovery that sometimes makes sense is to get O-Plan
to abandon its current activity and allow other parts of the system to
run.  You can accomplish this by selecting ``Return to pprocess
scheduler'' from the restart options presented after the Lisp error
message.  (This is equivalent to choosing "Return to scheduler" from
the menu appears when the error is first noticed.)

This approach should be tried only when the error occurs during a
relatively self-contained activity.  An example might be a Lisp error
when the PlanWorld Viewer tried to display a plan.  By returning
control the the scheduler, you could get the viewer to abandon this
attempt This would work fairly well because nothing else in the system
depends very strongly on the viewer succeeding.  On the other hand, it
would not work very well if, for instance, the {\sc dm} ran into an
error somewhere deep in the World State manager.  In such cases, you
might try typing the command (force-reset) instead.

\opsection{Control Panel and Trace Information within O-Plan}
The control panel allows the developer to set the levels of diagnostic
information and monitors within the system. Diagnostic levels can be
set in a range of values from {\tt none} through to {\tt all} where
all messages are printed. In most cases setting the level to {\tt
emergency only} is usually sufficient. The control panel also allows
the developer to place the system into single step mode in which the
developer is prompted on each agenda cycle to choose the next agenda
to be processed. A description of the single step mechanism is given
in step 6, (Agenda Manager Processing Options) of Section~\ref{demos}
which describes how to run an O-Plan demonstration.

The running lights process shows the developer which component is
currently running and in the case of a problem allows the developer to
easily identify in which component a problem has occurred. The running
lights window is controlled from a variable which allows the output to
be turned on and off. This can be achieved as follows:

To turn the running lights on (the default), have Lisp evaluate:

\begin{verbatim} 
  (run-lights-on)
\end{verbatim}

To turn the running lights off, have Lisp evaluate:

\begin{verbatim} 
  (run-lights-off)
\end{verbatim}

One way to get Lisp to evaluate these expressions is to type them in
the Lisp interaction window; another is to put them in the user's
``oplan-init'' file.  Note, however, that the oplan-init file is
examined only when O-Plan first starts up.

\opsection{\protect\label{demos} Example Demonstration}
The demonstrations allow the user to gain a gentle introduction to the
O-Plan systems through a series of simple worked examples. This
section describes the basic steps involved in running any of the
demonstrations.

The O-Plan system is menu driven (via the Task Assignment Window) which
allows the user to issue high level commands to the system. The Task
Assignment Window contains the following menu items:

\begin{verbatim}
    Status: Version <no> <date> <status of the planner> either:
            uninitialised, planner initialised, planning, 
            replanning, executing, planner finished
    Domain: one of none, <file> or <file> + <file>
    Task  : one of none or <task name>
    Authority: plan(all=inf), execute(all=no)

    * 1) Initialise Planner
      2) Input Task 
      3) Set Task
      4) Add to Task
      5) Plan View
      6) World View
      7) Replan
      8) Execute Plan
    * 9) Quit

    Please choose a number:- 
\end{verbatim}

In the Task Assignment Window items marked with a {\tt *} are the only
ones available from the planner at any point. A demonstration can be
broken down into ten distinct stages:

\begin{enumerate}
\item {\bf Initialising the System} \newline
Initialises the system ready for a new {\sc tf} domain description to be
accepted.

In order to carry out this step the following should be used:

\begin{quote} 
Move the mouse to the Task Assignment Window and enable the window for
keyboard input. Enter 1, (followed by carriage return) and this will
initialise the planner ready for the demonstration. Wait until the
planner initialised message appears in the {\tt Status} banner of the
Task Assignment Window.  
\end{quote}

\item {\bf Specifying the Domain} \newline
This part of the demonstration informs the planner of which domain is
to be used, e.g.  block stacking, house building, etc. 

In order to specify this information the following step should be
used:

\begin{quote}

Enter 2 from the Task Assignment Window. This will cause a further menu 
to appear which describes the problem domains available in the {\sc
\$oplandir}/{\sc tf} directory. Use the mouse pointer to either:

\begin{enumerate}
\item choose the file which contains the domain description you wish
to use.  
\item choose the {\tt Change directory} option. The user will be
presented with a menu listing the previous {\sc tf} directories
visited (where appropriate) and an option to type in a new directory
name. The user must either chose a directory from the menu or type the
name of a new directory after the prompt in the Task Assignment window.
The {\sc tf} files in the specified directory will now be displayed.
\item choose the {\tt Enter filename} option and type the name
of the file after the prompt in the Task Assignment window.
\end{enumerate}

Wait until the Task Assignment Window {\tt Status} banner displays the
message that is has successfully set up for the chosen domain. 
\end{quote}

\item {\bf Specifying the Task} \newline
This part of the demonstration informs the planner of the particular
task to be carried out within the domain. For example, the order of the
completed stack of blocks, the type of house, etc. 

In order to specify this information the following step should be
used:

\begin{quote}
Enter 3 from the Task Assignment Window. This will cause a new
menu to appear which describes the particular tasks in the domain you
may choose from (the list of task schemas in the {\sc tf} file). Use
the mouse pointer to choose the task you wish the planner to solve.  
\end{quote}

\item {\bf Adding a New Action to the Plan} \newline
This part of the demonstration informs the planner that the user
wishes to add further requirements to the plan. These can be added
once a plan has been generated and during the subsequent execution of
the plan. 

In order to specify this information the following step should be
used:

\begin{quote}
Enter 4 from the Task Assignment Window. This will cause a new menu
to appear which has a single option to ``Add an action''. Use the
mouse pointer to choose the option. The Task Assignment Window will
now be cleared to allow the input of the new action e.g. (evacuate
Calypso 20), (install kitchen\_equipment), etc. 
\end{quote}

\item {\bf Running the Planner on the Task} \newline 
This part of the demonstration usually involves little interaction
from the user apart from watching the messages from various windows.
However, the user is free to intervene as described in
Section~\ref{intervention} should they wish to though it is not
expected as part of a demonstration. As the planning processing moves
forward text will appear in each of the interface manager, agenda
manager, database manager and ks-platform windows respectively. The
text gives a trace of the planner as it solves the problem. It is not
necessary as part of the demonstration to understand the information
as it is being displayed.


\item  {\bf Messages and Diagnostics} \newline
The level of messages will be the default (normally short one line
messages). Greater or lesser levels of messages can be selected for
O-Plan using the Interface Manager's Control Panel. 

\item {\bf Agenda Manager Processing Options} \newline
The user may also control the order in which outstanding tasks (agenda
entries) are processed by the the O-Plan system. This is achieved by
placing the O-Plan system into single step mode. This is achieved by
chosing the {\tt Single Step} button from the Control Panel. This
will result in the following menu being displayed in the Agenda
Manager window.

Agenda Manager Interrogator: \newline
Stopped in cycle 61
\begin{enumerate}
\item {\tt Display the Agenda Table}: Display current list of
outstanding tasks
\item {\tt Display the Untriggered Agenda Table}: Display the current
list of untriggered agenda entries. 
\item {\tt Display the Alternatives Agenda Table}: Display the list of 
alternative plan state to explore
\item {\tt Process the top agenda entry}: Send the top entry for processing
\item {\tt Process any agenda entry}: Send a specified entry for
processing
\item {\tt Schedule KS-USER}: This will schedule the {\sc ks-user}
knowledge source to run on the next cycle of the planner. This will
allow the user to choose options from the menu described in
Section~\ref{intervention}.
\item {\tt Break in}: This allows the user to interrogate the contents
of particular datastructures within the {\sc lisp} code. You may
resume O-Plan by typing {\tt 0} 

\item {\tt Quit single step mode}
\end{enumerate}

\item {\bf Browsing on the Plan} \newline
Once the plan has been generated (or while it is being generated) by
the O-Plan system it can be viewed in a number of ways. In order to
specify the display mode Enter 5 from the Task Assignment Window.
This will invoke\footnote{Assuming the window is not already present}
the PlanWorld window which will display information from the planner
and allow the user to specify file names where necessary. 

The user will now be presented with a menu showing the different
viewing formats available. These are as follows:

\begin{enumerate}
\item {\tt Text to screen}: Send each node of the plan to the
PlanWorld window in the following format:

\begin{verbatim} 
   Node Number: node-3-10-2
      Begin_end Predecessors : (end_of node-3-9-1 begin_of node-3-10)
      Begin_end Successors   : (end_of node-3-10-2)
      End_end Predecessors   : (begin_of node-3-10-2)
      End_end Successors     : (begin_of node-3-9-8 begin_of node-3-10-3)
      Earliest_start_time    : 7~00:00:00
      Latest_start_time      : infinity
      Earliest_finish_time   : 9~00:00:00
      Latest_finish_time     : infinity
      Minimum_duration       : infinity
      Maximum_duration       : infinity
      Node_type              : action
      Node_pattern           : "(pour basement floor)"
\end{verbatim}

In most cases the size of the plan will cause information to be lost
off the top of the window and the user is advised to place the window
in {\tt scroll mode} buy pressing the middle mouse button and the
control key simultaneously and chosing the {\tt Enable Scrollbar}
option from the menu which appears.  

\item {\tt Text to File}: Send the plan to the file name specified by
the user and place it in the directory specified in the environment
variable {\sc \$oplantmpdir}. The user will be prompted for the file
name via the PlanWorld window. The format of each node in the file is
as follows:

\begin{verbatim} 
  node 
    <node number>
    (<begin end predecessors>)
    (<begin end successors>)
    (<end end predecessors>)
    (<end end successors>)
    (<earliest-start> <latest-start> <earliest-finish> <latest-finish> 
          <min-duration> <max-duration>)
    <node type>
    <node pattern>
  end_node
\end{verbatim}

An example of a node is as follows:

\begin{verbatim} 
  node
    node-3-4-1-2                                  
    (end_of node-3-4-1-1)                         
    (end_of node-3-4-1-2 begin_of node-3-4-1-2-1)
    (begin_of node-3-4-1-2 end_of node-3-4-1-2-1)
    (begin_of node-3-4-1-3)
    (0 1728000 172800 1900800 172800 1900800)
    action
    "(ground_move 107th_acr_%byair ft_meade_port andrews_afb_naf_afb)"
  end_node
\end{verbatim}

\item {\tt PostScript Graph}: Send the plan to a Postscript printer
for printing.  The user will be presented with a second menu from
which they must chose:
\begin{enumerate}
\item {\tt Single page}: Send a copy of the plan (scaled
to fit on a single sheet of A4) directly to a specified hardcopy
device. 
\item {\tt Multiple Pages}: Send a copy of the plan (split across
multiple sheets of A4) directly to a specified hardcopy
device. 
\end{enumerate}

Once the user has chosen their required output format a further menu
will appear indicating the number of levels. The user can either chose
to display the plan to a required level or display all nodes of the
plan by chosing the {\tt all} option.
\end{enumerate}

\item {\bf Browsing on Plan State Information} \newline
Further information concerning the effects asserted in the plan and
the goals satisfied during the generation of the plan, etc  can be
obtained through the Interface Manager's Control Panel. Two separate
browsing menus are provided:  

\paragraph{Interrogate the Contents of the DM}
To view this information chose the {\bf DM Developers Menu} option from
the Control Panel window. This will result in a new menu appearing in the
Database Manager window. The options in the menu are as follows:

\begin{enumerate}
\item {\tt Display nodes}: Display the nodes of the plan together with
their predecessors and successors.
\item {\tt Display TOME}: Display the effects asserted in the plan.
\item {\tt Display GOST}: Display the protected ranges over which
preconditions must be preserved.
\item {\tt Display PSV}: Display the plan state variables which are in
the plan.
\item {\tt Break in}: You may resume O-Plan by typing {\tt 0}.

\item Quit interrogation. Quit this menu
\end{enumerate}

\item {\bf Browsing on the World State} \newline
Information concerning the state of the world at a particular node in
the plan graph can be obtained and viewed on the screen. In order to
view this information Enter 6 from the Task Assignment Window. 
This will invoke\footnote{Assuming the window is not already present}
the PlanWorld window which will display information from the planner
and allow the user to specify file names where necessary. 

The user will be prompted for a node number by the phrase {\tt View at
end\_of node-} appearing in the Task Assignment Window. The user may
then select the plan node of interest. Following the choice of {\tt
node\_end} the user will be presented with a menu showing the
different viewing formats available. These are as follows:  
 
\begin{enumerate}
\item {\bf Text to Screen}: 
The information concerning the specified node of the plan will appear
in the {\tt PlanWorld} window as a set of triples of the form {\tt
$<$pattern$>$ = $<$value$>$}. For example:  

\begin{verbatim}
  {on a b} = true
  {clear a} = true
  {clear b} = false
\end{verbatim} 

\item {\bf Text to File}: Send the world view to the file name
specified by the user and place it in the directory specified in the
environment variable {\sc \$oplantmpdir}. The user will be prompted
for the file name via the PlanWorld window. 
\end{enumerate}
\end{enumerate}

\end{document}
.\" ------------------------- Change History--------------------------
.\"
.\" Modified to form part of the 2.2 Release: BD: June 29th 1994
.\" Modified to form part of the 2.3 Release: BD: March 14th 1995
.\" Modified to form part of the 3.1 Release: BD: Dec 11th 1996
.\"
.\" ------------------------------------------------------------------


.\" Local Variables:
.\" mode: latex
.\" version-command: "\\lastaltered"
.\" eval: (setq write-file-hooks (cons 'timestamp-for-latex write-file-hooks))
.\" End:
